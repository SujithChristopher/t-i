\documentclass{article}
\usepackage{layout}
\usepackage[utf8]{inputenc}
\usepackage[a4paper, total={6in, 8in}]{geometry}
\usepackage{graphicx}
\usepackage{amsmath}


\graphicspath{ {./src/} }

\begin{document}

\section*{Question 2}

\begin{center}
    \includegraphics*[scale=0.5]{lvdt.png}
\end{center}

Given the LVDT, the output $v_o(t)$ and input $v_p(t)$ voltages are given by:

\begin{equation*}
    v_o(t) = k . x . \sin(\omega_p t + \phi) 
    % \\v_p(t) = \sin(\omega_p t) 
\end{equation*}

\begin{equation*}
    v_p(t) = \sin(\omega_p t) 
\end{equation*}


And $L_p = 4mH, L_s = 2mH, R_p = 100 \Omega,
R_s = 50 \Omega$ find the output impedance and the frequency 
$\omega_p$ at which $\phi < 0.1\pi$\newline



To find the output impedance, we have to short the voltage sources and open the load resistance
\begin{center}
    \includegraphics*[scale=0.7]{ckt.png}
\end{center}

Output impedance is given by:
\begin{equation*}
    \mathcal{Z} = \sqrt{R_s^2  + X_L^2}
\end{equation*}

\begin{equation*}
    \mathcal{Z} = \sqrt{R_s^2  + (j\omega L_s)^2}
\end{equation*}

\begin{equation*}
    \mathcal{Z} = \sqrt{R_s^2  - \omega^2 L_s^2}
\end{equation*}

Therefore:
\begin{equation*}
    \phi = \tan^{-1}\frac{R_s}{\omega L_s}
\end{equation*}

Since we know that $\phi < 0.1\pi$, we can solve for $\omega_p$:
\begin{equation*}
    \omega_p < \frac{R_s}{L_s} \tan(0.1\pi)
\end{equation*}

\begin{equation*}
    \omega_p < 8122.99\, rad/s
\end{equation*}


\end{document}
