\documentclass{article}
\usepackage{layout}
\usepackage[utf8]{inputenc}
\usepackage[a4paper, total={6in, 10in}]{geometry}
\usepackage{graphicx}
\usepackage{amsmath}


\graphicspath{ {./src/} }
\begin{document}

\title{Assignment 6}
\date{}
\maketitle

\section*{Question 1}

Given:
\begin{itemize}
\item Cross section of the rectangle 20 mm x 5 mm
\item Two strain gauges attached along the axis and transverse 
direction $1231 \mu\, $strain and $-431\mu\, $strain
\end{itemize} 

Axial strain $ \varepsilon_a = 1231 \mu\, $strain,
Transversal strain $ \varepsilon_t = -431\mu\, $strain \\

Poison ratio is given by:
\begin{equation*}
    \nu    = -\frac{ \varepsilon_t}{\varepsilon_a}
\end{equation*}

\begin{equation*}
    \nu = -(\frac{-431 \mu }{1231 \mu })
\end{equation*}


\begin{equation*}
    % \frac{- \varepsilon_t}{\vareosilon_a} = \frac{1231}{-431}
    \nu = -(\frac{-431}{1231})
\end{equation*}

Poison's ratio is:
\begin{equation*}
    \nu    = 0.35
\end{equation*}


Stress applied on the strain gauge is given by:

\begin{equation*}
    \sigma = \frac{F}{A}
\end{equation*}

\begin{equation*}
    \sigma = \frac{400}{20 mm \times 5 mm}
\end{equation*}

Young's modulus is given by:

\begin{equation*}
    E = \frac{\sigma}{\varepsilon}
\end{equation*}

\begin{equation*}
    E = \frac{400}{20 * 5 * 10^{-6} * 1231 * 10^{-6}}
\end{equation*}


\begin{equation*}
    E = 3.25 G Pa
\end{equation*}


\section*{Question 2}

Given the following for a strain gauge:
\begin{itemize}
    \item Nominal resistance: $R = 150 \Omega$
    \item Gauge factor: $G = 2.5$
    \item $V_{in} = 3.3 V$
    \item $V_{out} = 10mV$
\end{itemize}

\begin{center}
    \includegraphics*[scale=0.5]{straingauge.png}
\end{center}

Whereas $R_1 = R_2 = R_3 = R_4 = R$

\begin{equation*}
    \frac{V_{out}}{V_{in}} = \frac{\Delta R}{ 4R + \Delta R} \, or \,  \frac{\Delta R}{ 4R}
\end{equation*}

\begin{equation*}
    \frac{10mV}{3.3V} = \frac{\Delta R}{ 4R}
\end{equation*}

\begin{equation*}
    \frac{10*10^{-3}}{3.3} = \frac{\Delta R}{ 4*150}
\end{equation*}

\begin{equation*}
    \Delta R = 1.81 \Omega
\end{equation*}

We know that $\frac{\Delta R}{R} = G \varepsilon $

\begin{equation*}
    \varepsilon = \frac{\Delta R}{GR}
\end{equation*}



Bringing to a common equation that can be used for other two cases:

\begin{equation*}
    \varepsilon = \frac{\Delta R}{2.5*150}
\end{equation*}

\begin{equation*}
    \varepsilon = \frac{\Delta R}{375}
\end{equation*}

We know that $\Delta R = 1.81 \Omega$

\begin{equation*}
    \varepsilon = \frac{1.81}{375}
\end{equation*}

\begin{equation*}
    \varepsilon = 0.0048
\end{equation*}

Case 2:

if two strain gauges are attached, such that the strain 
experienced by one strain gauge is $\varepsilon$ and the other is$ - \varepsilon$

\begin{equation*}
    \frac{V_{out}}{V_{in}} = \frac{\Delta R}{ 2R}
\end{equation*}

\begin{equation*}
    \frac{10*10^{-3}}{3.3} = \frac{\Delta R}{ 2*150}
\end{equation*}

\begin{equation*}
    \Delta R = 0.91 \Omega
\end{equation*}

Therefore the strain is given by:

\begin{equation*}
    \varepsilon = \frac{0.91}{375}
\end{equation*}

\begin{equation*}
    \varepsilon = 0.0024
\end{equation*}

Case 3:
if four strain gauges are attached, such that the strain
experienced by two strain gauge is $\varepsilon$ and the other two is$ - \varepsilon$

\begin{equation*}
    \frac{V_{out}}{V_{in}} = \frac{\Delta R}{ R}
\end{equation*}

\begin{equation*}
    \frac{10*10^{-3}}{3.3} = \frac{\Delta R}{ 150}
\end{equation*}

\begin{equation*}
    \Delta R = 0.45 \Omega
\end{equation*}

Therefore the strain is given by:

\begin{equation*}
    \varepsilon = \frac{0.45}{375}
\end{equation*}

\begin{equation*}
    \varepsilon = 0.0012
\end{equation*}

\end{document}
